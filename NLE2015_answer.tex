\documentclass[]{article}
\usepackage{color}
%opening
\title{Answer to decision on NLE-ARTC-15-0019 (Deep-neural network approaches for speech recognition with heterogeneous groups of speakers including children)}
\author{Romain Serizel, Diego Giuliani}

\begin{document}

\maketitle


\section{Reviewer 1}
\paragraph{Comments 1-3, 5-7, 9, 11, 13-14, 31}

~

This has been corrected
\paragraph{4. Page 7. 3rd equation. How is the quantity $p(X)$ calculated?}

~

$p(X)$ should actually be $p(S)$. This has been corrected.


\paragraph{8. Page 10, the penultimate paragraph, starting "The VTLN procedure..." is not clear. In the DNN that learns the warping
factor, are the inputs individual feature vectors or whole utterances? The figure suggests individual feature vectors, but the
sentence "Then training utterances and corresponding warping factors" suggests utterances. In the second part of this paragraph "This DNN is then used to produce the posterior probabilities of the VTLN warping factors..." needs to be clarified. Specifically you need to
say that the output of this DNN is a vector, and the dimension of the vector corresponds to the number of discrete VTLN normalisation
factors that are considered. This is clarified later, but it is confusing at this point.}

~

The paragraph has been modified into:

\textit{The VTLN procedure is first applied to generate a warping factor for each  utterance in the  training set. Each acoustic feature vector in the utterance is labelled with the utterance warping factor. Then, training acoustic feature vectors and corresponding warping factors are used to train a DNN classifier. Each class of the DNN correspond to one of the discrete VTLN factors and the dimension of the DNN output corresponds to the number of discrete VTLN factors. The DNN learns to infer the VTLN warping factor from the acoustic feature vector (Figure~1) or more precisely the posterior probability of each VTLN factors knowing the input acoustic feature vector. This DNN will be referred to as DNN-warp.}

\paragraph{10.1 Page 12. The first sentence is not grammatical and needs attention.}

~

The sentence has been modified into:

\textit{The ultimate goal here is not to estimate the VTLN warping factors but to perform robust speech recognition on heterogeneous corpora. To this end, the DNN-warp and the DNN-HMM can be optimised jointly (Figure~3).}

\paragraph{10.2 Also, there is more discussion of the "posteriors of the warping factor" and this has still not been explained, and the phrase "posteriors of the warping factor" suggests that there is only one warping factor.}

~

The posteriors probabilities are now explained p10 (see also comment 8). "Posterior probabilities of the warping factor" has been replaced everywhere by "Posterior probabilities of the warping factors" to avoid confusion.

\paragraph{12. Page 13, 4.1.1 given that VTLN is likely to be most effective for younger children, it would be informative to see the distribution of ages in the ChildIt corpus.}

~

A table reporting the age distribution has been added in Section 4.1.1 (see Table \ref{ChildItAge}).

\centering
\begin{table}
  \begin{minipage}{\textwidth}
\begin{tabular}{cccccccc}
\hline \hline
        & \multicolumn{7}{c} {Grade} \\
        & 2  &  3  & 4   & 5   & 6   & 7    & 8 \\ \hline
 N. Speakers       & 24 &  24 & 23  & 24  & 28  &  26  & 22 \\ \hline\hline
\end{tabular}
\end{minipage}
\caption{Distribution of speakers in the ChildIt corpus per grade. Children in grade 2 are approximatively 7 years old while children in grade 8 are approximatively 13 years old. \label{ChildItAge}}

\end{table}


\paragraph{13. 4.1.1 and 4.1.2 say something about the transcriptions of the data that are available - word-level or phone-level?}

~

It is now mentioned for each corpus only word-level transcription is available. Also, at the beginning of 4.2 we explain how the phone level transcription are obtained:

\textit{The approaches proposed in this paper have been first tested on small
corpora (ChildIt + APASCI) for phone recognition to explore as many
set-ups as possible in a limited amount of time.  The
reference phone transcription of an utterance was derived from the
corresponding word transcription by performing Viterbi decoding on a
pronunciation network.  This pronunciation network was built by
concatenation of the phonetic transcriptions of the words in the word
transcription.  In doing this alternative word pronunciations were
taken into account and an optional insertion of the silence model
between words was allowed.}

\paragraph{15. Page 15, 4.2.2. Please explain a little more about how the Hamming window is applied to the sequence of feature vectors. Is this normal in DNN training?}

~

See comment 16.

\paragraph{16. Page 15, 1 line later, what was the rationale for choosing DCT (on the previous page you applied HLDA)}

~

It is common to reduce the dimensionality of the input vector in DNN. DCT is probably not the most common choice (PCA and HLDA are commonly used) but it offers the advantage to be rather simple and requires little computational resources (as opposed to HLDA for example). Hamming windowing is applied to each frequency band (31-dimensional context in each band) before the DCT compression. This is a rather standard procedure. The processing description in the paper was not accurate, this has been corrected. Paragraph 4.2.2 has been corrected to:

\textit{The DNN  uses again  13 MFCC, including  the zero  order coefficient, computed on 20ms  frames with 10ms overlap. The context spans on a 31 frames  window. For each frequency band, the 31 coefficients context is separately scaled with Hamming and projected to a 16 dimensional vector using DCT. The 13 resulting vectors are concatenated to obtain a 208 dimensional feature  vector which is normalised to have zero-mean and unit variance before being used as input to the DNN. [\dots]}

Paragraph 4.5 has been modified to:

\textit{The DNN-warp inputs are the MFCC with a 61 frames context window, DCT projected to a 208 dimensional features vector (the procedure is similar as in 4.2.2)}

\paragraph{17. Page 15, 2nd and 3rd paragraphs. How was the size of the network chosen? How were all of the parameters in DNN training chosen? Is performance sensitive to the precise values of these parameters?}

~

These are rather standard parameters when using DNN for acoustic modelling in automatic speech recognition: context is usually between 9 frames and 40 frames, hidden layers size between 1000 and 2000 elements and the DNN usually have 3 to 8 hidden layers. We simply adjusted these values to our specific corpus. For example the corpus is rather small therefore we have only 4 hidden layers to prevent over-fitting The performance are sensitive to these parameters but they will not be affected drastically by minor changes on the hyper-parameters value. This is now mentioned at the end of the 2nd paragraph in 4.2.2:

\textit{The values of the hyper-parameters (network topology and learning parameters) are standard values, in the range of the values commonly used for these parameters in the literature.} 

\paragraph{18. Page 16. 3rd line. Please explain what is meant by "... single consonants and their germinant counterparts..." I have no idea what this means.}

~

In phonetics, gemination or consonant elongation happens when a spoken
consonant is  pronounced for an audibly  longer period of  time than a
short  consonant. Gemination is  distinct from  stress and  may appear
independently of  it. Gemination literally means  ``twinning'', and is
from the same Latin root as ``Gemini''.

Consonant length  is distinctive in some  languages including Italian.
In  these  languages most  consonants  have  two  versions: the  short
(``single'') version  and the corresponding long  version (its ``geminate                                                 
counterpart'').

For ASR  purposes all  consonants are modelled,  however, traditionally
for  the Italian  language when  computing the  phone error  rate ``no                                                    
distinction  is  made between  single  consonants  and their  geminate                                                    
counterparts''. The  motivation is  that the main  acoustic difference
between ``single  consonants and their geminate  counterparts'' is the
duration  so that  a single  consonant is  highly confusable  with its
geminate counterpart.  To avoid to inflate  phone recognition results,
confusion  between a  consonant and  its geminate  counterpart  is not
considered as an error.

We have  not changed the  text in the  paper as it  seems sufficiently
clear that  the phone error rate is  computed on a reduced  set of 28
phone labels.


\paragraph{19. Page 16, 4.3. Does the HMM set used for word recognition use the same set of tied states as the previous HMM set?}

~

The set of tied states are different. This is now mentioned explicitly.\\
Paragraph 4.2.1:

\textit{Acoustic models  are 3039 tied-state triphone HMM based on a set of 48 phonetic units derived from the SAMPA Italian alphabet.}
\\
Paragraph 4.3.1:

\textit{The HMM-GMM are similar to those used for phone recognition except that they use more Gaussian densities to benefit from the extensive training data. Acoustic models  are 5021 tied-state triphone HMM based on a set of  48 phonetic units  derived from the  SAMPA Italian alphabet. Each tied-state is modelled with  a mixture of  32 Gaussian densities   having  a  diagonal   covariance  matrix. In  addition, ``silence''  is  modelled  with  a  Gaussian mixture  model  having  32 Gaussian densities.}
\paragraph{20. line 2. Be more precise. What exactly do you mean by saying that the DNN was trained on a different set of Gaussians? This is too imprecise.}

~

This has been corrected, paragraph 4.3.2 is now:

\textit{The DNN  are similar to those  used for phone  recognition except that they are  trained on a different  set of targets. The  targets of the DNN are the 5021 tied-states obtained from the word recognition HMM-GMM training on the mixture of adults' and children's  speech (ChildIt + IBN). The DNN has 4 hidden  layers, each of which  contains 1500 elements  such that the DNN architecture  can be summarised  as follows: 208  x 1500 x  1500 x 1500 x 1500 x 5021.}

\paragraph{21. Page 17, 4.4. In adaptation are all of the other DNN training parameters the same as in 4.2.2}

~

Yes all the training parameters are similar to 4.2. Paragraph 4.4 is now:

\textit{One option is to adapt an already trained general DNN  to group specific corpora. The data architecture is the same as described above. The initial DNN weights are the weights obtained with a pre-training/training procedure applied on all training data available  (ChildIt+APASCI, respectively ChildIt + IBN). The DNN is then trained with back propagation on a group specific corpora (ChildIt, adult female speech in APASCI and adult male speech in APASCI, respectively IBN). The training parameters are the same as during the general training (4.2.2 and 4.3.2, respectively) and the learning rate follows the same rule as above. The mini-batch size is 512 and a first-order momentum of 0.5 is applied.}

\paragraph{22. Page 17, last line. This is the first time that the number of VTLN factors is specified, or even that it is acknowledged that only a discrete set of factors is considered and therefore it is possible to create a vector of posterior probabilities. This basic principle needs to be introduced much earlier to understand how the DNN for estimating VTLN factor posteriors works.}

~

This is now mentioned more explicitly in paragraph 3.1: 

\textit{[\dots] A  well  known  method  for estimating  the scaling  factor is \textbf{ based on  a  grid search  over a  discrete set  of possible scaling  factors} by maximizing the likelihood  of warped data given  a  current set  of  HMM-based acoustic  models~(Lee and Rose, 1996). Frequency scaling  is performed by  warping the power  spectrum during signal  analysis  or, for  filter-bank  based  acoustic front-end,  by changing the  spacing and width  of the filters while  maintaining the spectrum  unchanged~(Lee and Rose, 1996). In  this  work we  adopted  the latter approach \textbf{considering a discrete set of VTLN factors}. Details on the VTLN implementation  are provided in Section~4.5.}

Also an explanation about the link between VTLN factors and posterior probabilities has been been given as answer to comment 8.

\paragraph{23. Page 18. 1st paragraph. Why do you use HMMs with just 1 Gaussian component per state to optimise the VTLN factor? }

~

The   approach  for  estimating   the  VTLN   scaling factors  making   use  of
speaker-independent  triphone  HMMs  with  just 1 Gaussian density  per  state  was
proposed by Welling et al. in the reference paper:
\begin{itemize}
\item L. Welling, S. Kanthak and  H.~Ney, ``Improved Methods for Vocal Tract                                              
Normalization'', in Proc. of IEEE ICASSP, 1999, Vol. 2, pp. 761-764.
\end{itemize}

In the paper it is empirically verified that having 1 Gaussian density
per state is a good  choice when compared
with the use of many Gaussian densities per state.

In past works, we  adopted this approach  in the context  of children's
and adults' speech recognition  with good results,  see for example  the reference
papers:

\begin{itemize}
\item M. Gerosa, D. Giuliani  and Fabio Brugnara, ``Acoustic variability and
automatic recognition of children's speech'', Speech Communication,
2007, Vol. 49, N. 10-11, pp. 847-860.
\item M. Gerosa, D. Giuliani and Fabio Brugnara, ``Towards age-independent acoustic modeling'',
Speech Communication, 2009, Vol. 51, N. 6, pp. 499-509.
\end{itemize}


\paragraph{24. Page 18, paragraph 3. In the specification of the DNN, what is the meaning of the (5021) in brackets? Does this mean that there are different numbers of tied states in different systems? If so, please explain.}

~

The paragraph has been modified into:

\textit{[\dots] The new DNN acoustic model  has 4 hidden layers,
each of  which contains 1500  elements such that the  DNN architecture
can then be summarized  as follows: 233 x 1500 x 1500  x 1500 x 1500 x
3039 for phone recognition and 233 x 1500 x 1500  x 1500 x 1500 x 5021 for word recognition.}

\paragraph{25. Page 18. Where does the learning rate of 0.0002 come from?}

~

The learning rate was chosen empirically. For higher learning rates, the training accuracy would improve but not the cross-validation accuracy. The DNN obtained were not saved by the training script until both training and cross-validation accuracy progressed (at learning rate 0.0002). Setting the learning rate directly at 0.0002 is just a way to speed up the training process. Paragraph 4.6 is now:

\textit{The DNN-warp and DNN-HMM can be fine-tuned jointly with back-propagation. In such case, the starting learning rate is set to 0.0002 in the first 4 hidden layers (corresponding to the DNN-warp) and to 0.0001 in the last 4 hidden layers (corresponding to the DNN-HMM). The learning rate is chosen empirically as the highest value for which both training accuracy and cross-validation accuracy improve. Setting a different learning rate in the first 4 hidden layers and the last 4 hidden layers is done in a attempt to overcome the vanishing gradient effect in the 8 layers DNN obtained from the concatenation of the DNN-warp and the DNN-HMM. The learning rates are then adapted following the same schedule as described above. The joint optimisation is done with a modified version of the TNet software package~(Vesely et al., 2010).}

\paragraph{26. A general point. The adapted system is close to optimal in all cases, but most of the discussion is biased towards the benefits of VTLN. For me, one of the most interesting consequences of this paper is that adaptation appears to compensate for VTL differences without any explicit VTLN. Of course, one of the problems with the "adapted system" is that model selection is not done automatically. A simple way to achieve this would be to compute the probability of an utterance for all three adapted models and then apply the highest scoring one to recognition. Why didn't you do something like this? It may be that even if this approach makes an "error" and, say, classifies a particular child as a female adult, the female adult model may give best performance on this child's data.}

~

This comment is partially answered in comment 29. The solution you propose for model selection seems indeed interesting. Yet, our goal in this paper was to experiment on features and acoustic modelling with DNN and we voluntarily decided to limit the scope of the paper on these particular points for clarity. The approach you mention involves a full decoding system and is therefore beyond the scope of this paper. In the introduction:

%This is now clearly stated in the abstract:
%
%\textit{For clarity sake, the scope of this paper is voluntarily limited to approaches based only on acoustic modelling and features transform based on VTLN.}

\textit{This paper is a proof of concept and the authors voluntarily limited its focus to simple ASR systems. Only  age and gender groups problems and approaches based only on acoustic modeling and VTLN transform are considered here in order to focus on the effects of these particular approaches. Therefore state-of-the-art approaches based on speaker identity models such as I-vectors (Dehak, Kenny, Dehak, Dumouchel, and Ouellet, 2011; Saon, Soltau, Nahamoo, and Picheny, 2013; Senior and Lopez-Moreno, 2014), speaker codes (Abdel-Hamid and Jiang, 2013a), linear input networks and linear output networks (Li and Sim, 2010) are beyond the scope of this paper.} % At this initial stage, the authors decided to focus on approaches based only on acoustic modelling therefore excluding approaches requiring an entire decoding stream (or at least a language model), such as approaches based on confusion networks (Mangu, Brill, and Stolcke, 2000).}

~

And in the conclusion:

\textit{This paper voluntarily focused on age and gender groups problems and on approaches based only on acoustic modeling and VTLN based transform in order to focus on the effects of these particular approaches. Extension of this work could consider approaches based on speaker identity models such as I-vectors, speaker codes, linear input networks and linear output networks.} % and approaches, such as system combination based on confusion networks, that require an entire decoding stream (or at least a language model).}
\paragraph{27. Page 22. In the earlier overview of the systems, did you include the case of MFCC feature vectors augmented with VTLN warping factors obtained in the normal way?}

~

We did try to augment the MFCC with the warping factors obtained in the standard way the results of these experiments are reported in Table 5 row {\em Warp + MFCC}.

\paragraph{28. Page 22. Exactly how are the posterior probabilities averaged to utterance level, and why is this done?}

~

We compute a vector of averaged posterior probability for each warping factor over utterances. This can be considered as an intermediate step between the standard way to obtain warping factors and using the DNN-warp. This way we can check what is the impact of having a hard or soft decision on the warping factors ({\em Warp + MFCC} vs. {\em Warp-post (utt) + MFCC}) without considering the effect of the time unit on which the warping factor are computed. The effect of the time unit are checked with: {\em Warp-post + MFCC} vs. {\em Warp-post (utt) + MFCC}. This is now clarified in an additional paragraph after the 1st paragraph of 5.1.3:

\textit{To compute the vectors {\em Warp-post (utt) + MFCC} the posterior probability of each warping factor is averaged over utterances to obtain a vector of averaged posterior probabilities. This experiment allow to study independently the effects of having a soft or hard decision on the warping factor selection and the effects of the time unit used to compute the warping factors. The impact of having of hard or soft decision on the warping factors is studied comparing {\em Warp + MFCC} to {\em Warp-post (utt) + MFCC}. While the effects of the time unit used to compute warping factors are studied comparing {\em Warp-post + MFCC} to {\em Warp-post (utt) + MFCC}}

\paragraph{29. Page 23. Final paragraph. This is an example of what I was indicating earlier. The first line of the final paragraph is very "pro-VTLN" and ignores the fact that better performance is obtained by simple adaptation.}

~

We do agree with the reviewer comment on this particular sentence and the second half of the sentence has been removed to try to be less biased. However, we believe that the last paragraph before the conclusion and the conclusion itself are not particularly biased towards VTLN approach. Indeed we do mention that adaptation generally outperform VTLN approach, that they both have their advantages and drawbacks and that, ultimately, if you want the best performance you should combine these approaches (if you can allow it computationally).

\paragraph{30. Just a comment, but this seems a lot of effort to obtain a VTLN-based system that outperforms adaptation/model selection, and for children the best performance is obtained with model selection augmented with VTLN!}

~

See previous comment.

\paragraph{32. Page 28. Lines 5-7. Augmenting features that have already been VTLN normalised with the posterior probabilities of the VTLN factors seems an odd thing to do. What is the motivation/justification?}

~

VTLN-normalisation operates at utterances level whereas posterior probabilities are obtained at frame level. While estimating VTLN factors on a longer time unit (utterance) should allow for a more accurate average estimation, the "true" warping factor might be fluctuating in time (Miguel, Lieida, Rose, Buera, and Ortega, 2005; Maragakis and Potamianos,
2008). We believe that combining VTLN normalisation at utterance level and posterior probabilities estimated at frame level should help overcoming this problem. According to results, this seems to be true. A second paragraph stating this has been added in 5.2.2:

\textit{The approach combining VTLN-normalised features and posterior probabilities aims at testing the complementary between VTLN-normalisation that operates at utterances level and posterior probabilities that are obtained at frame level. While estimating VTLN factors on a longer time unit (utterance) should allow for a more accurate average estimation, the "true" warping factor might be fluctuating in time (Miguel, Lieida, Rose, Buera, and Ortega, 2005; Maragakis and Potamianos,
2008). Combining VTLN normalisation at utterance level and posterior probabilities estimated at frame level should help overcoming this problem.}


\section{Reviewer 2}

\paragraph{1. In section 4.5, the authors conduct a grid search based on a set of 25 warping factors via the standard maximum likelihood using a triphone single Gaussian system. My question is that are there any difficulties in using a more robust triphone GMM system to search the warping factors? Single Gaussian is definitely not robust enough for a reliable warping factor. The easiest argument is that at least two Gaussians are needed to model both genders. In addition, what if the warping factor is per speaker rather than per utterance?}

~

The   approach  for  estimating   the  VTLN   scaling factors  making   use  of
speaker-independent  triphone  HMMs  with  just 1 Gaussian density  per  state  was
proposed by Welling et al. in the reference paper:
\begin{itemize}
\item L. Welling, S. Kanthak and  H.~Ney, ``Improved Methods for Vocal Tract                                              
Normalization'', in Proc. of IEEE ICASSP, 1999, Vol. 2, pp. 761-764.
\end{itemize}

In the paper it is empirically verified that having 1 Gaussian density
per state is a good  choice when compared
with the use of many Gaussian densities per state.

In past works, we  adopted this approach  in the context  of children's
and adults' speech recognition  with good results,  see for example  the reference
papers:

\begin{itemize}
\item M. Gerosa, D. Giuliani  and Fabio Brugnara, ``Acoustic variability and
automatic recognition of children's speech'', Speech Communication,
2007, Vol. 49, N. 10-11, pp. 847-860.
\item M. Gerosa, D. Giuliani and Fabio Brugnara, ``Towards age-independent acoustic modeling'',
Speech Communication, 2009, Vol. 51, N. 6, pp. 499-509.
\end{itemize}

Although VTLN scaling factor selection on a speaker-by-speaker basis
could be more robust, in this work we adopted a per utterance approach
as this approach better matches with a realistic application scenario
in which only little data is presented for decoding.


\paragraph{2. The speaker adaptation is achieved by retraining of an age/gender independent DNN with an age/gender specific corpus “adaptation”. How does this approach compare with the stateoftheart DNN adaptation schemes? For example, using ivectors or speaker codes as additional inputs to the DNN? What if the speakeradapted features (e.g., fMLLR) are used as the DNN inputs?}

~

I-vectors and speaker codes as additional input should probably perform better than the approaches presented here (and especially speaker codes). Yet, they require speaker annotations which is out of the scope of the paper. For clarity sake, we decided to focus in a first stage on the age and gender groups problem which requires only annotations about the age class (adult or children) and the gender class (male or female). These annotations are far more simple than speaker annotations required by the aforementioned approaches. However, in the future it would indeed be interesting to compare the approaches presented here and speaker based approaches. All this is clarified in introduction and conclusion (see also the answer to comment 26 from reviewer 1).

Regarding speaker adapted features such as fMLLR, they are widely used as input to DNN in many ASR tools available. However, at this stage the target of this paper is to compare VTLN to approaches based only on DNN (that could be inspired by VTLN) not to compare several features normalisation approaching which has already been done. Yet, we agree that in later stage it would be interesting to compare results obtained with our best approach to a wider range of normalisation or adaptation techniques.

\paragraph{3. How is system combination performed? I am not sure I understand what the authors mean by “features level” in section 5.1.4. More implementation details are appreciated. How about the recognition level combination schemes, e.g., confusion network combination? Bayesian risk decoding? etc.}

~

We are not really doing system combination in the classical sense. What we do is combining the different approaches we have introduced in a single system. Therefore, standard systems combination approaches such as those based on confusion networks are out of the scope of this paper. To avoid the confusion we replaced any reference to "systems combination" by "combination of different approaches". We also modified the paragraph about the combination of different approaches as follows:

\textit{Combining several approaches is a common way to improve systems performance and robustness. It is decided here to combine the different approaches introduced until here to exploit their potential complementarity. In ASR, systems are generally combined using confusion networks or using late fusion at transcription level. The solution chosen here is different. It was chosen to either combine the different approaches at features level (standard VTLN normalised features and the posterior probabilities of the warping factors are combined at the input of the DNN) or at the acoustic model level (acoustic features augmented with the posterior probabilities of the warping factors are used as input to a DNN with age-gender adaptation). A reason to this choice is that this way, the experiments are focused at acoustic model level and remain independent from any change in the later stage of the decoder or in the language model.}



\section{Reviewer 3}

\paragraph{Comments III-V}

~

This has been corrected

\paragraph{I. Why the baseline DNN system is not sequence trained., eg. bMMI, sMBR which is now standard DNN based acoustic modeling techniques and should be used as baselines.}

~

Sequence trained systems have indeed proved really efficient during the past years. However, as explained above (see also comment 26 from reviewer 1 and comment 2 from reviewer 2) we decided to keep the system in its most simple form in order to emphasis the impact of the approaches that are proposed. This paper should been see as a proof of concept and we believe it is better to keep the core system simple to deliver a clear message. 

Yet, we totally agree with you on the fact that in a more advanced stage the system should include some kind of sequence training and an analysis of its effects on different age and gender groups. But once again this is not in the scope of this paper.
\paragraph{II. In the under-resourced conditions, the authors should also investigate more adaptation techniques, e.g., regularization, LIN., LON., etc. to see if they perform better. I would like to see more experimental results on more DNN based adaptation approaches.}

~

See comment 26 from reviewer 1 and comment 2 from reviewer 2.




\end{document}
